\documentclass[11pt, a4paper]{article}

%%%%%%%%%%%%%%%%%%%%%%%%% Credit %%%%%%%%%%%%%%%%%%%%%%%%

% template ini dibuat oleh martin.manullang@if.itera.ac.id untuk dipergunakan oleh seluruh sivitas akademik itera.

%%%%%%%%%%%%%%%%%%%%%%%%% PACKAGE starts HERE %%%%%%%%%%%%%%%%%%%%%%%%
\usepackage{graphicx}
\usepackage{caption}
\usepackage{microtype}
\captionsetup[table]{name=Tabel}
\captionsetup[figure]{name=Gambar}
\usepackage{tabulary}
\usepackage{minted}
\usepackage{amsmath}
\usepackage{fancyhdr}
\usepackage{amssymb}
\usepackage{amsthm}
\usepackage{placeins}
\usepackage{amsfonts}
\usepackage{graphicx}
\usepackage[all]{xy}
\usepackage{tikz}
\usepackage{verbatim}
\usepackage[left=2cm,right=2cm,top=3cm,bottom=2.5cm]{geometry}
\usepackage{hyperref}
\hypersetup{
    colorlinks,
    linkcolor={red!50!black},
    citecolor={blue!50!black},
    urlcolor={blue!80!black}
}
\usepackage{caption}
\usepackage{subcaption}
\usepackage{multirow}
\usepackage{psfrag}
\usepackage[T1]{fontenc}
\usepackage[scaled]{beramono}
% Enable inserting code into the document
\usepackage{listings}
\usepackage{xcolor} 
% custom color & style for listing
\definecolor{codegreen}{rgb}{0,0.6,0}
\definecolor{codegray}{rgb}{0.5,0.5,0.5}
\definecolor{codepurple}{rgb}{0.58,0,0.82}
\definecolor{backcolour}{rgb}{0.95,0.95,0.92}
\definecolor{LightGray}{gray}{0.9}
\lstdefinestyle{mystyle}{
	backgroundcolor=\color{backcolour},   
	commentstyle=\color{green},
	keywordstyle=\color{codegreen},
	numberstyle=\tiny\color{codegray},
	stringstyle=\color{codepurple},
	basicstyle=\ttfamily\footnotesize,
	breakatwhitespace=false,         
	breaklines=true,                 
	captionpos=b,                    
	keepspaces=true,                 
	numbers=left,                    
	numbersep=5pt,                  
	showspaces=false,                
	showstringspaces=false,
	showtabs=false,                  
	tabsize=2
}
\lstset{style=mystyle}
\renewcommand{\lstlistingname}{Kode}
%%%%%%%%%%%%%%%%%%%%%%%%% PACKAGE ends HERE %%%%%%%%%%%%%%%%%%%%%%%%


%%%%%%%%%%%%%%%%%%%%%%%%% Data Diri %%%%%%%%%%%%%%%%%%%%%%%%
\newcommand{\student}{\textbf{Nashwa Putri Laisya (122140180)}}
\newcommand{\course}{\textbf{Sistem Teknologi Multimedia (IF25-40305)}}
\newcommand{\assignment}{\textbf{Worksheet 1: Setup Python Environment untuk Multimedia}}

%%%%%%%%%%%%%%%%%%% using theorem style %%%%%%%%%%%%%%%%%%%%
\newtheorem{thm}{Theorem}
\newtheorem{lem}[thm]{Lemma}
\newtheorem{defn}[thm]{Definition}
\newtheorem{exa}[thm]{Example}
\newtheorem{rem}[thm]{Remark}
\newtheorem{coro}[thm]{Corollary}
\newtheorem{quest}{Question}[section]
%%%%%%%%%%%%%%%%%%%%%%%%%%%%%%%%%%%%%%%%
\usepackage{lipsum}%% a garbage package you don't need except to create examples.
\usepackage{fancyhdr}
\pagestyle{fancy}
\lhead{Nashwa Putri Laisya (122140180)}
\rhead{ \thepage}
\cfoot{\textbf{Worksheet 1: Setup Python Environment untuk Multimedia}}
\renewcommand{\headrulewidth}{0.4pt}
\renewcommand{\footrulewidth}{0.4pt}

%%%%%%%%%%%%%%  Shortcut for usual set of numbers  %%%%%%%%%%%

\newcommand{\N}{\mathbb{N}}
\newcommand{\Z}{\mathbb{Z}}
\newcommand{\Q}{\mathbb{Q}}
\newcommand{\R}{\mathbb{R}}
\newcommand{\C}{\mathbb{C}}
\setlength\headheight{14pt}

%%%%%%%%%%%%%%%%%%%%%%%%%%%%%%%%%%%%%%%%%%%%%%%%%%%%%%%555
\begin{document}
\thispagestyle{empty}
\begin{center}
	\includegraphics[scale = 0.15]{Figure/ifitera-header.png}
	\vspace{0.1cm}
\end{center}
\noindent
\rule{17cm}{0.2cm}\\[0.3cm]
Nama: \student \hfill Tugas Ke: \assignment\\[0.1cm]
Mata Kuliah: \course \hfill Tanggal: \today\\
\rule{17cm}{0.05cm}
\vspace{0.1cm}



%%%%%%%%%%%%%%%%%%%%%%%%%%%%%%%%%%%%%%%%%%%%% BODY DOCUMENT %%%%%%%%%%%%%%%%%%%%%%%%%%%%%%%%%%%%%%%%%%%%%
\section{Tujuan Pembelajaran}
Setelah menyelesaikan worksheet ini, mahasiswa diharapkan mampu:
\begin{itemize}
    \item Memahami pentingnya manajemen environment Python untuk pengembangan multimedia
    \item Menginstall dan mengkonfigurasi Python environment menggunakan conda, venv, atau uv
    \item Menginstall library-library Python yang diperlukan untuk multimedia processing
    \item Memverifikasi instalasi dengan mengimpor dan menguji library multimedia
    \item Mendokumentasikan proses konfigurasi dan hasil pengujian dalam format \LaTeX
\end{itemize}

\section{Latar Belakang}
Python telah menjadi bahasa pemrograman yang sangat populer untuk multimedia processing karena memiliki ekosistem library yang sangat kaya. Namun, untuk dapat bekerja dengan multimedia secara efektif, kita perlu mengatur environment Python dengan benar dan menginstall library-library yang tepat.

Manajemen environment Python sangat penting untuk:
\begin{itemize}
    \item Menghindari konflik antar library (dependency conflict)
    \item Memastikan reproducibility dari project
    \item Memudahkan kolaborasi antar developer
    \item Memisahkan project yang berbeda dengan requirement yang berbeda
\end{itemize}

\section{Instruksi Tugas}

\subsection{Persiapan}
\textbf{Sebelum memulai, pastikan Anda telah:}
\begin{itemize}
    \item Menginstall Python 3.8 atau lebih baru di sistem Anda
    \item Memilih salah satu tool manajemen environment: \textbf{conda}, \textbf{venv}, atau \textbf{uv}
    \item Membuka terminal/command prompt
    \item Menyiapkan dokumen \LaTeX\ ini untuk dokumentasi
\end{itemize}

\subsection{Bagian 1: Membuat Environment Python}
Pilih \textbf{SALAH SATU} dari tiga opsi berikut dan ikuti langkah-langkahnya.

\subsubsection{Opsi 1: Menggunakan Conda (Direkomendasikan untuk pemula)}
Jalankan perintah berikut di terminal:

\begin{lstlisting}[language=bash, caption=Membuat environment dengan Conda]
# Membuat environment baru dengan nama 'multimedia'
conda create -n multimedia python=3.11

# Mengaktifkan environment
conda activate multimedia

# Verifikasi environment aktif
conda info --envs
\end{lstlisting}

\subsubsection{Opsi 2: Menggunakan venv (Built-in Python)}
\begin{lstlisting}[language=bash, caption=Membuat environment dengan venv]
# Membuat environment baru
python3 -m venv multimedia-env

# Mengaktifkan environment (Linux/Mac)
source multimedia-env/bin/activate

# Mengaktifkan environment (Windows)
# multimedia-env\Scripts\activate

# Verifikasi environment aktif
which python
\end{lstlisting}

\subsubsection{Opsi 3: Menggunakan uv (Modern dan cepat)}
\begin{lstlisting}[language=bash, caption=Membuat environment dengan uv]
# Install uv terlebih dahulu jika belum ada
# pip install uv

# Membuat environment baru
uv venv multimedia-uv

# Mengaktifkan environment (Linux/Mac)
source multimedia-uv/bin/activate

# Mengaktifkan environment (Windows)
# multimedia-uv\Scripts\activate

# Verifikasi environment aktif
which python
\end{lstlisting}

\textbf{Dokumentasi:}
\begin{itemize}
    \item Tool manajemen environment yang saya pilih: \textbf{uv}
    \item Screenshot atau copy-paste output dari perintah verifikasi environment
    \begin{figure}
        \centering
        \includegraphics[width=0.9\textwidth]{Figure/install_uv.png}
        \caption{Screenshot pembuatan environment uv}
        \label{fig:instalasi_uv}
    \end{figure}
\end{itemize}

\subsection{Bagian 2: Instalasi Library Multimedia}
Setelah environment aktif, install library-library berikut:

\subsubsection{Library Audio Processing}
\begin{lstlisting}[language=bash, caption=Instalasi library audio]
# Untuk conda:
conda install -c conda-forge librosa soundfile scipy

# Untuk pip (venv/uv):
pip install librosa soundfile scipy
\end{lstlisting}

\subsubsection{Library Image Processing}
\begin{lstlisting}[language=bash, caption=Instalasi library image]
# Untuk conda:
conda install -c conda-forge opencv pillow scikit-image matplotlib

# Untuk pip (venv/uv):
pip install opencv-python pillow scikit-image matplotlib
\end{lstlisting}

\subsubsection{Library Video Processing}
\begin{lstlisting}[language=bash, caption=Instalasi library video]
# Untuk conda:
conda install -c conda-forge ffmpeg
pip install moviepy

# Untuk pip (venv/uv):
pip install moviepy
\end{lstlisting}

\subsubsection{Library General Purpose}
\begin{lstlisting}[language=bash, caption=Instalasi library umum]
# Untuk conda:
conda install numpy pandas jupyter

# Untuk pip (venv/uv):
pip install numpy pandas jupyter
\end{lstlisting}

\newpage
\textbf{Dokumentasi Instalasi Library Audio Processing:}
\begin{itemize}
    \item Perintah instalasi yang saya gunakan
    \begin{lstlisting} [language=Python, caption=Perintah instalasi library audio processing yang digunakan, label=cmdiAudio]
        python -m pip install librosa soundfile scipy \end{lstlisting}
    \item Screenshot proses instalasi atau output sukses
    \begin{figure}[h]
        \centering
        \includegraphics[width=0.4\textwidth]{Figure/install_library_audio.png}
        \caption{Screenshot instalasi library audio processing}
        \label{fig:instalasi_audio}
    \end{figure}
    \newpage
    \item Daftar library yang berhasil diinstall dengan versinya
    \begin{figure}[h]
        \centering
        \includegraphics[width=0.9\textwidth]{Figure/audio_sukses.png}
        \caption{Library audio processing yang berhasil diinstal}
    \end{figure}
\end{itemize}

\textbf{Dokumentasi Instalasi Library Image Processing:}
\begin{itemize}
    \item Perintah instalasi yang saya gunakan
    \begin{lstlisting} [language=Python, caption=Perintah instalasi library image processing yang digunakan, label=cmdiImage]
        python -m pip install opencv-python pillow scikit-image matplotlib \end{lstlisting}
    \item Screenshot proses instalasi atau output sukses
    \begin{figure}[h]
        \centering
        \includegraphics[width=0.8\textwidth]{Figure/install_library_image.png}
        \caption{Screenshot instalasi library image processing}
        \label{fig:instalasi_image}
    \end{figure}
    \newpage
    \item Daftar library yang berhasil diinstall dengan versinya
    \begin{figure}[h]
        \centering
        \includegraphics[width=0.9\textwidth]{Figure/sukses_image.png}
        \caption{Library image processing yang berhasil diinstal}
    \end{figure}
\end{itemize}

\textbf{Dokumentasi Instalasi Library Video Processing:}
\begin{itemize}
    \item Perintah instalasi yang saya gunakan
    \begin{lstlisting} [language=Python, caption=Perintah instalasi library video processing yang digunakan, label=cmdiVideo]
        python -m pip install moviepy \end{lstlisting}
    \item Screenshot proses instalasi atau output sukses
    \begin{figure}[h]
        \centering
        \includegraphics[width=0.8\textwidth]{Figure/install_library_video.png}
        \caption{Screenshot instalasi library video processing}
        \label{fig:instalasi_video}
    \end{figure}
    \item Daftar library yang berhasil diinstall dengan versinya
    \begin{figure}[h]
        \centering
        \includegraphics[width=0.9\textwidth]{Figure/video_sukses.png}
        \caption{Library video processing yang berhasil diinstal}
    \end{figure}
\end{itemize}

\textbf{Dokumentasi Instalasi Library General Purpose:}
\begin{itemize}
    \item Perintah instalasi yang saya gunakan
    \begin{lstlisting} [language=Python, caption=Perintah instalasi library general purpose yang digunakan, label=cmdiGP]
        python -m pip install numpy pandas jupyter \end{lstlisting}
    \item Screenshot proses instalasi atau output sukses
    \begin{figure}[h]
        \centering
        \includegraphics[width=0.4\textwidth]{Figure/install_library_general1.png}
        \caption{Screenshot instalasi library general purpose}
        \label{fig:instalasi_general}
    \end{figure}
    \newpage
    \item Daftar library yang berhasil diinstall dengan versinya
    \begin{figure}[h]
        \centering
        \includegraphics[width=0.9\textwidth]{Figure/general_sukses.png}
        \caption{Library general processing yang berhasil diinstal}
    \end{figure}
\end{itemize}

\subsection{Bagian 3: Verifikasi Instalasi}
Buat file Python sederhana untuk menguji semua library yang telah diinstall

\textbf{Jalankan script dan dokumentasikan hasilnya}

\subsection{Bagian 4: Simple Test dengan Sample Code}
Buat dan jalankan contoh sederhana untuk setiap kategori multimedia:

\subsubsection{Test Audio Processing}
\begin{lstlisting}[language=Python, caption=Test audio processing sederhana]
import numpy as np
import matplotlib.pyplot as plt

# Generate simple sine wave
duration = 2  # seconds
sample_rate = 44100
frequency = 440  # A4 note

t = np.linspace(0, duration, int(sample_rate * duration))
audio_signal = np.sin(2 * np.pi * frequency * t)

# Plot waveform
plt.figure(figsize=(10, 4))
plt.plot(t[:1000], audio_signal[:1000])  # Plot first 1000 samples
plt.title('Sine Wave (440 Hz)')
plt.xlabel('Time (s)')
plt.ylabel('Amplitude')
plt.grid(True)
plt.savefig('sine_wave_test.png', dpi=150, bbox_inches='tight')
plt.show()

print(f"Generated {duration}s sine wave at {frequency}Hz")
print(f"Sample rate: {sample_rate}Hz")
print(f"Total samples: {len(audio_signal)}")
\end{lstlisting}

\subsubsection{Test Image Processing}
\begin{lstlisting}[language=Python, caption=Test image processing sederhana]
import numpy as np
import matplotlib.pyplot as plt
from PIL import Image

# Create a simple test image
width, height = 400, 300
image = np.zeros((height, width, 3), dtype=np.uint8)

# Add some patterns
image[:, :width//3, 0] = 255  # Red section
image[:, width//3:2*width//3, 1] = 255  # Green section
image[:, 2*width//3:, 2] = 255  # Blue section

# Add a white circle in the center
center_x, center_y = width//2, height//2
radius = 50
Y, X = np.ogrid[:height, :width]
mask = (X - center_x)**2 + (Y - center_y)**2 <= radius**2
image[mask] = [255, 255, 255]

# Display and save
plt.figure(figsize=(8, 6))
plt.imshow(image)
plt.title('Test Image with RGB Stripes and White Circle')
plt.axis('off')
plt.savefig('test_image.png', dpi=150, bbox_inches='tight')
plt.show()

print(f"Created test image: {width}x{height} pixels")
print(f"Image shape: {image.shape}")
print(f"Image dtype: {image.dtype}")
\end{lstlisting}

\textbf{Dokumentasikan hasil eksekusi:}
\begin{itemize}
    \item Screenshot output dari kedua script di atas
    \item Gambar yang dihasilkan
    \item Error message jika ada dan cara mengatasinya
\end{itemize}

\section{Bagian Laporan}

\subsection{Output Verifikasi Instalasi}
\textbf{Perintah yang digunakan:}
\begin{lstlisting}[language=Python, caption=Perintah untuk verifikasi instalasi]
try:
    import librosa
    print("librosa: Success")
except Exception as e:
    print("librosa: ERROR ->", e)

try:
    import soundfile
    print("soundfile: Success")
except Exception as e:
    print("soundfile: ERROR ->", e)

try:
    import scipy
    print("scipy: Success")
except Exception as e:
    print("scipy: ERROR ->", e)

try:
    import cv2
    print("opencv (cv2): Success")
except Exception as e:
    print("opencv: ERROR ->", e)

try:
    from PIL import Image
    print("Pillow (PIL): Success")
except Exception as e:
    print("Pillow: ERROR ->", e)

try:
    import skimage
    print("scikit-image: Success")
except Exception as e:
    print("scikit-image: ERROR ->", e)

try:
    import matplotlib
    print("matplotlib: Success")
except Exception as e:
    print("matplotlib: ERROR ->", e)

try:
    import moviepy
    print("moviepy: Success")
except Exception as e:
    print("moviepy: ERROR ->", e)

try:
    import numpy
    print("numpy: Success")
except Exception as e:
    print("numpy: ERROR ->", e)

try:
    import pandas
    print("pandas: Success")
except Exception as e:
    print("pandas: ERROR ->", e)
\end{lstlisting}

\textbf{Output}
\begin{lstlisting}[caption={Output verifikasi instalasi}]
(multimedia-uv) nashwalaisya@192 ~ % python "/Users/nashwalaisya/Documents/Semester - 7/Sistem Teknologi Multimedia/Worksheet 1/verifikasi_instalasi.py"
librosa: Success
soundfile: Success
scipy: Success
opencv (cv2): Success
Pillow (PIL): Success
scikit-image: Success
matplotlib: Success
moviepy: Success
numpy: Success
pandas: Success
\end{lstlisting}

\subsection{Screenshot Hasil Test}
\begin{itemize}
    \item Terminal/command prompt yang menunjukkan environment aktif
    \begin{figure}[h]
        \centering
        \includegraphics[width=0.8\textwidth]{Figure/envir_aktif.png}
        \caption{Aktivasi environment}
        \label{activate-environment}
    \end{figure}
    \item Output dari script test audio (sine wave plot)
    \begin{figure}[h]
        \centering
        \includegraphics[width=0.8\textwidth]{Figure/testing_audio.png}
        \caption{Output script test audio pada terminal}
        \label{sine_wave_plot}
    \end{figure}
    \begin{figure}[h]
        \centering
        \includegraphics[width=0.8\textwidth]{Figure/sine_wave_test.png}
        \caption{Output script test audio}
        \label{sine_wave_plot}
    \end{figure}
    \newpage
    \item Output dari script test image (RGB stripes dengan circle)
    \begin{figure}[h]
        \centering
        \includegraphics[width=0.8\textwidth]{Figure/testing_image.png}
        \caption{Output script test image pada terminal}
        \label{rgbstripes}
    \end{figure}
    \begin{figure}[h]
        \centering
        \includegraphics[width=0.5\textwidth]{Figure/test_image.png}
        \caption{Output script test image}
        \label{rgbstripes}
    \end{figure}
\end{itemize}

\subsection{Analisis dan Refleksi}
\textbf{Jawab pertanyaan berikut:}

\begin{enumerate}
    \item \textbf{Mengapa penting menggunakan environment terpisah untuk project multimedia?}
    
    \textit{Jawaban saya: Environment penting karena dapat memengaruhi kinerja teknis, pengujian, hingga pengalaman pengguna yang mana hal-hal tersebut sangat krusial untuk project multimedia.}
    
    \item \textbf{Apa perbedaan utama antara conda, venv, dan uv? Mengapa Anda memilih tool yang Anda gunakan?}
    
    \textit{Jawaban saya: Perbedaan utama antara conda, venv, dan uv adalah:
    \begin{itemize}
        \item conda -> reliable tapi berat
        \item venv -> tidak terlalu berat tapi kecepatannya tergantung pip
        \item uv -> paling cepat di antara ketiga environment
    \end{itemize}
    Saya memilih uv karena tidak berat untuk laptop saya dan juga tidak perlu menunggu lama untuk menginstallnya.}
    
    \item \textbf{Library mana yang paling sulit diinstall dan mengapa?}
    
    \textit{Jawaban saya: Selama penginstallan saya tidak menemukan kesulitan untuk menginstall library mana pun. Mungkin ada salah satu library yang memakan waktu lebih lama dari pada library-library lainnya. Library tersebut adalah Library General Purpose (numpy pandas jupyter). Proses penginstallan memang memakan waktu yang sedikit leboh lama tetapi tidak ada kesulitan dalam prosesnya.}
    
    \item \textbf{Bagaimana cara mengatasi masalah dependency conflict jika terjadi?}
    
    \textit{Jawaban saya: Seandainya terjadi dependency conflict, biasanya masalah itu timbul karena lebih dari 1 package memerlukan versi yang tidak kompatibel dari dependency yang sama. Jadi lebih baik buat environment baru, lalu tetapkan versi secara eksplisit dan instal semuanya sekaligus kemudian verifikasi.}
    
    \item \textbf{Jelaskan fungsi dari masing-masing library yang berhasil Anda install!}
    
    \textit{Jawaban saya:}
    \begin{itemize}
        \item \textit{librosa -> untuk analisis musik dan audio: memuat file audio, menghitung spektrogram, mengekstrak fitur (tempo, pitch, MFCC), deteksi beat, dll.}
        \item \textit{soundfile -> untuk membaca dan menulis file suara.}
        \item \textit{scipy -> library komputasi ilmiah umum, dengan submodul scipy.signal untuk pemrosesan sinyal: penyaringan (filtering) audio, transformasi Fourier, interpolasi.}
        \item \textit{opencv-python (cv2) -> library computer vision untuk pemrosesan gambar/video secara real-time: transformasi gambar, deteksi wajah, pelacakan objek, deteksi tepi.}
        \item \textit{Pillow (PIL) -> untuk manipulasi gambar dasar: membuka, memotong, mengubah ukuran, memutar, mengonversi format (JPEG, PNG, dll.).}
        \item \textit{scikit-image (skimage) -> toolkit pemrosesan gambar tingkat lanjut berbasis NumPy/SciPy: segmentasi, denoising, morfologi, ekstraksi fitur.}
        \item \textit{matplotlib -> untuk plotting dan visualisasi: menampilkan gambar, membuat histogram, menggambar kurva data.}
        \item \textit{moviepy -> pengeditan video di Python: memotong, menggabungkan, menambah efek, menambahkan audio, membuat GIF.}
        \item \textit{ffmpeg (alat backend, tidak di-import langsung) -> program command-line yang kuat untuk mengonversi dan memproses video/audio.}
        \item \textit{numpy -> library inti untuk komputasi numerik di Python: operasi cepat pada array/matriks, aljabar linear, FFT.}
        \item \textit{pandas -> untuk analisis dan manipulasi data: membaca CSV/Excel, mengelola data tabular (baris & kolom), grup, filter.}
        \item \textit{jupyter -> menyediakan environment Jupyter Notebook: pemrograman interaktif, menjalankan Python per cell, menampilkan grafik langsung di notebook.}
    \end{itemize}
\end{enumerate}

\section{Export Environment untuk Reproduksi}
Sebagai langkah terakhir, export environment Anda agar dapat direproduksi:

\subsection{Untuk Conda}
\begin{lstlisting}[language=bash, caption=Export conda environment]
conda env export > environment.yml
\end{lstlisting}

\subsection{Untuk venv/uv}
\begin{lstlisting}[language=bash, caption=Export pip requirements]
pip freeze > requirements.txt
\end{lstlisting}

\textbf{Isi file requirements.txt:}

\begin{lstlisting}[caption=Environment/Requirements file]
anyio==4.10.0
appnope==0.1.4
argon2-cffi==25.1.0
argon2-cffi-bindings==25.1.0
arrow==1.3.0
asttokens==3.0.0
async-lru==2.0.5
attrs==25.3.0
audioop-lts==0.2.2
audioread==3.0.1
babel==2.17.0
beautifulsoup4==4.13.5
bleach==6.2.0
certifi==2025.8.3
cffi==1.17.1
charset-normalizer==3.4.3
comm==0.2.3
contourpy==1.3.3
cycler==0.12.1
debugpy==1.8.16
decorator==5.2.1
defusedxml==0.7.1
executing==2.2.0
fastjsonschema==2.21.2
fonttools==4.59.2
fqdn==1.5.1
h11==0.16.0
httpcore==1.0.9
httpx==0.28.1
idna==3.10
imageio==2.37.0
imageio-ffmpeg==0.6.0
ipykernel==6.30.1
ipython==9.4.0
ipython_pygments_lexers==1.1.1
ipywidgets==8.1.7
isoduration==20.11.0
jedi==0.19.2
Jinja2==3.1.6
joblib==1.5.2
json5==0.12.1
jsonpointer==3.0.0
jsonschema==4.25.1
jsonschema-specifications==2025.4.1
jupyter==1.1.1
jupyter-console==6.6.3
jupyter-events==0.12.0
jupyter-lsp==2.3.0
jupyter_client==8.6.3
jupyter_core==5.8.1
jupyter_server==2.17.0
jupyter_server_terminals==0.5.3
jupyterlab==4.4.6
jupyterlab_pygments==0.3.0
jupyterlab_server==2.27.3
jupyterlab_widgets==3.0.15
kiwisolver==1.4.9
lark==1.2.2
lazy_loader==0.4
librosa==0.11.0
llvmlite==0.44.0
MarkupSafe==3.0.2
matplotlib==3.10.5
matplotlib-inline==0.1.7
mistune==3.1.3
moviepy==2.2.1
msgpack==1.1.1
nbclient==0.10.2
nbconvert==7.16.6
nbformat==5.10.4
nest-asyncio==1.6.0
networkx==3.5
notebook==7.4.5
notebook_shim==0.2.4
numba==0.61.2
numpy==2.2.6
opencv-python==4.12.0.88
packaging==25.0
pandas==2.3.2
pandocfilters==1.5.1
parso==0.8.5
pexpect==4.9.0
pillow==11.3.0
platformdirs==4.4.0
pooch==1.8.2
proglog==0.1.12
prometheus_client==0.22.1
prompt_toolkit==3.0.52
psutil==7.0.0
ptyprocess==0.7.0
pure_eval==0.2.3
pycparser==2.22
Pygments==2.19.2
pyparsing==3.2.3
python-dateutil==2.9.0.post0
python-dotenv==1.1.1
python-json-logger==3.3.0
pytz==2025.2
PyYAML==6.0.2
pyzmq==27.0.2
referencing==0.36.2
requests==2.32.5
rfc3339-validator==0.1.4
rfc3986-validator==0.1.1
rfc3987-syntax==1.1.0
rpds-py==0.27.1
scikit-image==0.25.2
scikit-learn==1.7.1
scipy==1.16.1
Send2Trash==1.8.3
setuptools==80.9.0
six==1.17.0
sniffio==1.3.1
soundfile==0.13.1
soupsieve==2.8
soxr==0.5.0.post1
stack-data==0.6.3
standard-aifc==3.13.0
standard-chunk==3.13.0
standard-sunau==3.13.0
terminado==0.18.1
threadpoolctl==3.6.0
tifffile==2025.8.28
tinycss2==1.4.0
tornado==6.5.2
tqdm==4.67.1
traitlets==5.14.3
types-python-dateutil==2.9.0.20250822
typing_extensions==4.15.0
tzdata==2025.2
uri-template==1.3.0
urllib3==2.5.0
wcwidth==0.2.13
webcolors==24.11.1
webencodings==0.5.1
websocket-client==1.8.0
wheel==0.45.1
widgetsnbextension==4.0.14
\end{lstlisting}

\section{Kesimpulan}
\textbf{Dari saya:}
\begin{itemize}
    \item Pengalaman setup Python environment untuk multimedia -> Awalnya saya kesulitan buat install environment karena saya belum terbiasa dengan command di macOS, tapi setelah tau command di macOS, saya jadi mudah untuk install environmentnya dan untuk commandnya kurang lebih sama seperti OS lainnya, hanya saja ada tambahan " python -m " sebelum menggunakan pip.
    \item Persiapan untuk project multimedia selanjutnya -> Mungkin saya akan coba untuk cari-cari gimana gambaran untuk audio, image, dan video processing. Setidaknya saya bisa tau workflownya.
    \item Saran untuk mahasiswa lain yang akan melakukan setup serupa -> Mungkin gunakan environment yang compatible dengan masing-masing device saja, jangan terlalu dipaksakan untuk pakai yang mungkin tidak memungkinkan untuk devicenya.
\end{itemize}

\textit{Environment itu penting adanya untuk project multimedia. Kenapa? Karena dengan adanya environment, performance teknisnya jadi lebih smooth dan pengujian project juga jadi lebih baik dan sesuai}

\newpage
\section{Referensi}
\begin{enumerate}
    \item \href{https://chatgpt.com/share/68b02dd3-9fb4-8004-834e-b273df17297a}{Pertanyaan seputar environment - ChatGPT}
    \item \href{https://chatgpt.com/share/68b02e33-969c-8004-bfe0-5c48a221b90b}{Membuat file Python sederhana untuk verifikasi instalasi - ChatGPT}
\end{enumerate}

\newpage
\bibliographystyle{IEEEtran}
\bibliography{Referensi}
\end{document}